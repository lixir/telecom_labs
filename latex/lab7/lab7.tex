\documentclass[a4paper]{article}
\usepackage[utf8x]{inputenc}
\usepackage[T1,T2A]{fontenc}
\usepackage[russian]{babel}
\usepackage{hyperref}
\usepackage{indentfirst}
\usepackage{listings}
\usepackage{color}
\usepackage{here}
\usepackage{array}
\usepackage{multirow}
\usepackage{graphicx}
\usepackage{amsmath} 
\usepackage{spverbatim}

\usepackage{caption}
\renewcommand{\lstlistingname}{Программа} % заголовок листингов кода

\lstset{ %
	extendedchars=\true,
	keepspaces=true,
	language=bash,					% choose the language of the code
	basicstyle=\footnotesize,		% the size of the fonts that are used for the code
	numbers=left,					% where to put the line-numbers
	numberstyle=\footnotesize,		% the size of the fonts that are used for the line-numbers
	stepnumber=1,					% the step between two line-numbers. If it is 1 each line will be numbered
	numbersep=5pt,					% how far the line-numbers are from the code
	backgroundcolor=\color{white},	% choose the background color. You must add \usepackage{color}
	showspaces=false				% show spaces adding particular underscores
	showstringspaces=false,			% underline spaces within strings
	showtabs=false,					% show tabs within strings adding particular underscores
	frame=single,           		% adds a frame around the code
	tabsize=2,						% sets default tabsize to 2 spaces
	captionpos=b,					% sets the caption-position to bottom
	breaklines=true,				% sets automatic line breaking
	breakatwhitespace=false,		% sets if automatic breaks should only happen at whitespace
	escapeinside={\%*}{*)},			% if you want to add a comment within your code
	postbreak=\raisebox{0ex}[0ex][0ex]{\ensuremath{\color{red}\hookrightarrow\space}}
}

\usepackage[left=2cm,right=2cm,
top=2cm,bottom=2cm,bindingoffset=0cm]{geometry}

\begin{document}	% начало документа

\begin{titlepage}	% начало титульной страницы

	\begin{center}		% выравнивание по центру

		\largeФедеральное государственное автономное образовательное учреждение высшего образования «Санкт-Петербургский политехнический университет Петра Великого» \\
		\large Институт компьютерных наук и технологий \\
		\large Кафедра компьютерных систем и программных технологий\\[2cm]
		% название института, затем отступ 6см
		
	    \vfill
		\hugeТелекоммуникационные технологии\\[0.5cm] % название работы, затем отступ 0,5см
		\large Лабораторная работа №7:\\
		Помехоустойчивое кодирование\\[4.8cm]

	\end{center}

	\begin{flushright} % выравнивание по правому краю
		\begin{minipage}{0.25\textwidth} % врезка в половину ширины текста
			\begin{flushleft} % выровнять её содержимое по левому краю

				\large\textbf{Работу выполнил:}\\
				\large Сергеев ~А.А.\\
				\large {Группа:} 33531/2\\
				
				\large \textbf{Преподаватель:}\\
				\large Богач ~Н.В.\\

			\end{flushleft}
		\end{minipage}
	\end{flushright}
	
	\vfill % заполнить всё доступное ниже пространство

	\begin{center}
	\large Санкт-Петербург\\
	\large \the\year % вывести дату
	\end{center} % закончить выравнивание по центру

\thispagestyle{empty} % не нумеровать страницу
\end{titlepage} % конец титульной страницы
\vfill % заполнить всё доступное ниже пространство

% Содержание
\tableofcontents
\newpage
\section{Цель}
Изучение методов помехоустойчивого кодирования и сравнение их свойств.
\section{Постановка задачи}
\begin{enumerate}
    \item Провести кодирование/декодирование сигнала, полученного с помощью функции $randerr$ кодом Хэмминга $2$-мя способами: с помощью встроенных функций $encode$/$decode$, а также через создание проверочной и генераторной матриц и вычисление синдрома. Оценить корректирующую способность кода.
    \item Выполнить кодирование/декодирование циклическим кодом, кодом БЧХ, кодом Рида-Соломона. Оценить корректирующую способность кода.
\end{enumerate}
\section{Теоретический раздел}
Функция encode/decode осуществляют кодирование и декодирование соответственно сообщения с использованием блочного кода. Тип используемого кода задаётся в числе параметров функции. Линейный блочный код в общем случае описывается порождающей матрицей (generator matrix). Кодирование блока (вектора) производится путём его умножения на порождающую матрицу. При контроле ошибок на приёмной стороне используется проверочная матрица кода (parity-check matrix). Преобразование порождающей матрицы в проверочную и обратно осуществляется функцией $gen2par$. Если умножение кодированного блока на проверочную матрицу не даёт нулевого вектора, то полученный результат (его называют синдром -- syndrome) позволяет определить, какие именно символы были искажены в процессе передачи. Для двоичного кода это позволяет исправить ошибки. Декодирование линейного блочного кода, таким образом, можно осуществить с помощью таблицы, в которой для каждого значения синдрома указан соответствующий вектор ошибок. Создать такую таблицу на основании проверочной матрицы кода позволяет функция $syndtable$. Функция $gfweight$ позволяет определить кодовое расстояние для линейного блочного кода по его порождающей или проверочной матрице.\\
\subsection{Циклические коды}
Циклические коды -- это подкласс линейных кодов, обладающие тем свойством, что циклическая перестановка символов в кодированном блоке даёт другое возможное кодовое слово того же рода. Для работы с циклическими кодами в пакете $Communications$ есть две функции. Задав число символов в кодируемом и закодированном блоках, с помощью функции $cyclpoly$ можно получить порождающий полином циклического кода. Далее, использовав этот полином в качестве одного из параметров функции $cyclgen$, можно получить порождающую и проверочную матрицы для данного кода. 
\subsection{Коды БЧХ}
Коды БЧХ являются одним из подклассов циклических блочных кодов. Для работы с ними функции высокого уровня вызывают специализированные функции $bchenc$ (кодирование) и $bchdec$ (декодирование). Кроме того, функция $bchgenpoly$ позволяет рассчитывать параметры или порождающий полином для двоичных кодов БЧХ.
\subsection{Коды Хэмминга}
Коды Хэмминга являются одним из подклассов циклических блочных кодов. Порождающий полином для кодов Хэмминга неприводим и примитивен, а длина кодированного блока равна $2m-1$. Порождающая и проверочная матрицы для кодов Хэмминга генерируются функцией $hammgen$.
\subsection{Коды Рида-Соломона}
Коды Рида-Соломона являются одним из подклассов циклических блочных кодов. Это единственные поддерживаемые пакетом $Communications$ недвоичные коды. Для работы с кодами Рида-Соломона функции высокого уровня вызывают специализированные функции $rsenc$(кодирование) и $rsdec$(декодирование). Кроме того, функции $rsenc$ и $rsdec$ осуществляют кодирование и декодирование текстового файла. Функция $rsgenpoly$ генерирует порождающие полиномы для кодов Рида-Соломона.
\section{Ход работы}
\subsection{Код Хэмминга}
\subsubsection{Decode/Encode}
Генерируем сообщение функцией $randerr$ длиной $11$, проводим его кодирование и декодирование с использованием функций $encode$ и $decode$:
\lstinputlisting[language=Matlab]{lab7/dec_enc.m}\\
Сгенерированное сообщение: 
$0 0 1 0$.\\
Закодированное сообщение:
$ 1 1 1 0 0 1 0$.\\
Закодированное сообщение с 1 ошибкой:
$1 1 0 0 0 1 0$.\\
Декодированное сообщение:
$0 0 1 0$.\\
Как видно, изначально сгенерированное сообщение совпадает с декодированным.\\
Теперь допустим в закодированном сообщении две ошибки и попробуем провести декодирование:\\

Сгенерированное сообщение:
$0 0 1 0$.\\
Закодированное сообщение:
$ 1 1 1 0 0 1 0$.\\
Закодированное сообщение с 2мя ошибками:
$1 1 0 1 0 1 0$.\\
Декодированное сообщение:
$1 0 0 0$.\\
ERROR \\
\subsubsection{Кодирование с помощью проверочной и генераторной матриц}
Произведем кодирование/декодирование сигнала кодом Хэмминга с помощью проверочной и генераторной матриц и вычислим синдром.\\
При умножении исходного сообщения на генераторную матрицу в ее конечной части.\\
сохраняется исходная посылка, т.к. соответствующий блок генераторной матрицы представляет собой единичную матрицу. Оставшуюся часть формирует контрольные биты.\\
Формирование синдрома происходит с помощью домножения на проверочную матрицу.\\
Синдром указывает на ошибочный бит в посылке. Далее, он исправляется.
\lstinputlisting[language=Matlab]{lab7/use_matrix.m}\\

msg = $0 0 1 0$.\\
m = $1 1 1 0 0 1 0$.\\
Допускаем ошибку: \\
m =$1 0 1 0 0 1 0$.\\
synd = $0 1 0$.\\
В переводе в десятиричную систему счисления получаем 2.\\
z = $0, 1, 0, 0, 0, 0, 0$.\\
Исправленное сообщение без ошибки:
$1 1 1 0 0 1 0$.\\
Корректирующая способность кода равна $1$.
\subsection{Циклический код}
Производим кодирование и декодирование сообщения:
\lstinputlisting[language=Matlab]{lab7/cycle_code.m}\\

Сгенерированное сообщение: $0 0 1 0$.\\
pol = $1 1 0 1$\\
code = $1 1 1 0 0 1 0$\\
code = $1 0 1 0 0 1 0$\\
synd = $2 1 2$\\
synd = $0 1 0$\\
stbl =\\
	 0     0     0     0     0     0     0\\
     0     0     1     0     0     0     0\\
     0     1     0     0     0     0     0\\
     0     0     0     0     1     0     0\\
     1     0     0     0     0     0     0\\
     0     0     0     0     0     0     1\\
     0     0     0     1     0     0     0\\
     0     0     0     0     0     1     0\\
\\
syndrde2 = 2\\
z = $0 1 0 0 0 0 0$\\
rez = $1 1 1 0 0 1 0$\\
Был построен полином циклического кода $x^3 + x^2 + 1$
который использовался в качестве параметра функции cyclgen. Получены порождающая и проверочная матрицы.Корректирующая способность кода равна $1$.
\subsection{Код БЧХ}
\lstinputlisting[language=Matlab]{lab7/bch.m}\\

Сгенерированное сообщение: $0 0 1 0$.\\
Закодированное сообщение (без ошибки): $0 0 1 0 1 1 0$.\\
Закодированное сообщение (с ошибкой): $0 1 1 0 1 1 0$.\\
Декодированное сообщение: $0 0 1 0$.\\
Нетрудно убедиться, что исходное и полученное сообщения идентичны, что свидетельствует об успешном исправлении допущенной ошибки.\\
Корректирующая способность кода равна $1$.
\section{Вывод}
В данной лабораторной работе проведены кодирования и декодирования посылок кодами Хэмминга, циклическим и БЧХ. Выбирать метод кодирования стоит в зависимости от типа посылки и зашумленности канала. Код Хэмминга достаточно простой в использовании и не требует больших мощностей. Но существенным недостаткой является то, что все рассмотренные в работе коды позволяют исправить только одну ошибку в бинарном коде.
\end{document}